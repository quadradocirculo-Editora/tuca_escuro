\raggedright

{\Formular{\textbf{TUCA VIEIRA}}} {\Formular{\textit{é fotógrafo, jornalista, pesquisador e artista visual. Com graduação em Letras (\scalebox{.8}{FFLCH-USP}, 1998) e mestrado em Arquitetura e Urbanismo (\scalebox{.8}{FAU-USP}, 2018), tem como eixo central de trabalho a cidade e suas representações. Trabalha como fotógrafo profissional desde 1991, tendo atuado no jornal {\slsc{Folha de S. Paulo}} (2002–2008), onde realizou dezenas de reportagens, sobretudo na capital paulista. É desta época a famosa fotografia intitulada {\slsc{Paraisópolis}} (2004), uma das mais conhecidas imagens do Brasil contemporâneo, que se tornou símbolo da desigualdade social ao redor do mundo.
Participou de diversas exposições e é colaborador de publicações no Brasil e no exterior. Possui fotografias nas coleções Pirelli/\scalebox{.8}{MASP}, Pinacoteca do Estado de São Paulo, Instituto Moreira Salles e Instituto Itaú Cultural, entre outras instituições públicas e privadas. Foi vencedor do {\slsc{Prêmio Funarte de Arte Contemporânea}} (2013), do {\slsc{Prêmio Porto Seguro de Fotografia}} (2010) e do {\slsc{Prêmio Folha de Jornalismo}} (2013), entre outros. É também autor do {\slsc{Atlas Fotográfico da cidade de São Paulo e seus arredores}} – uma série de 203 fotografias em grande formato, numa tentativa exaustiva de representação da maior cidade brasileira, ganhador do {\slsc{Prêmio \scalebox{.8}{APCA} de arquitetura – categoria pesquisa}} (2017). Atualmente desenvolve o projeto {\slsc{Hipercidades}}, que pretende percorrer as 30 cidades do mundo com mais de 10 milhões de habitantes.}}}




